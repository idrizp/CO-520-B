% \begin{center}
%     \begin{circuitikz}
%         % Draw a voltage source from 0 to 5V
%         \draw (0,0) to [sqV, voltage dir=RP, v=$V_{in}$] (0,4);
%         \draw (0,4) to [R, l=$100\Omega$] (\textwidth/4,4);
%         \draw (\textwidth/4,4) to [L, l=$10\text{mH}$] (\textwidth/2,4);
%         % Draw a C that connects to the voltage in
%         \draw (\textwidth/2,4) to [C, l=$6\text{n}8\text{F}$] (\textwidth/2,0);
%         \draw (\textwidth/2,0) to (0,0);
%     \end{circuitikz}
% \end{center}

\begin{center}
    \begin{tabular}{cc}
        % Left Column (Text)
        \begin{minipage}{0.4\linewidth}
            \begin{itemize}
                \item $V_{pp} = 1\text{V}$
                \item $V_{off} = 0.5\text{V}$
                \item $f = 100\text{Hz}$
                \item $R_i = 50\Omega$
            \end{itemize}
        \end{minipage}
         &
        % Right Column (Circuit)
        \begin{minipage}{0.6\linewidth}
            \begin{circuitikz}
                % Draw a voltage source from 0 to 5V
                \draw (0,0) to [sqV, voltage dir=RP, v=$V_{in}$] (0,4);
                \draw (0,4) to [R, l=$100\Omega$] (\linewidth/4,4);
                \draw (\linewidth/4,4) to [L, l=$10\text{mH}$] (\linewidth/2,4);
                % Draw a C that connects to the voltage in
                \draw (\linewidth/2,4) to [C, l=$6\text{n}8\text{F}$] (\linewidth/2,0);
                \draw (\linewidth/2,0) to (0,0);
            \end{circuitikz}
        \end{minipage}
    \end{tabular}
\end{center}

\begin{enumerate}
    \item The function generator was set to produce a 100Hz square wave with an amplitude of $0.5\text{V}$ and an offset of $0.5\text{V}$. It was checked with the oscilloscope if the signal modulated between $0\text{V}$ and $1\text{V}$.

    \item Subsequently, the R-decade was set to $100\Omega$, and the oscilloscope was connected in parallel to the capacitor.

    \item The damped frequency $f_d$ was measured. To determine $f_d$, the time or frequency of the exponentially damped sinusoidal waveform was measured. A hardcopy was taken of one signal period and another focusing on the ringing phenomenon.

    \item Afterward, the damped radian frequency $\omega_d$ was calculated. In this calculation, the internal resistance of the function generator was considered to be $50\Omega$. The calculated value was compared with the measured value from step (2). If they were consistent, the next steps were proceeded with.

    \item The resistance required for the circuit to be critically damped was then calculated. The signal was displayed, and a hardcopy was taken.

    \item To check if the practical signal was critically damped, the R-decade value was varied, and a hardcopy of the final result was taken.

    \item Finally, the R-decade was set to $30\text{k}\Omega$, causing the circuit to be over-damped. The transient voltage across the capacitor was displayed, and a hardcopy was taken.
\end{enumerate}


In this experiment, the transient response of the RLC circuit to various different configurations of resistance was explored.

The oscilloscope, in general, showed what is expected of the circuit. The transient voltage of the capacitor was observed to show the ringing phenomenon characteristic of underdamped systems due to the low resistance of $100\Omega$.

When the resistance was tuned to close to the optimal nominal values for a critically damped response, the transient response was observed to be slightly overdamped. This is because of the R-Decade having its own internal resistance, as well as the internal resistance of the non-ideal capacitor.

The resistance was then tuned until critical damping behaviour was observed. The value has a slight error from the nominal value, for reasons discussed above. Afterward, the resistance was increased to $30k\Omega$ to view the effect of overdamping, where the result was shown in the oscilloscope, as expected.

To solve the evaluation circuits, second order differential equations were used. The damping behaviour was generally determined by solving for the damping ratio, $\zeta$, and the general solution to the circuit was then solved by finding the initial conditions $C_1$ and $C_2$, and by finding the particular solution.
\section{Problem 1: AM Modulated Signals in Time Domain}
The AM signal is generated by the function generator, where a sinusoid with a modulation index of 70\%, a carrier frequency of 20kHz, and a voltage of 10$V_{pp}$ is generated. The frequency and amplitude properties of the signal are then measured.

\begin{figure}[H]
    \centering
    \includegraphics[width=0.75\textwidth]{images/execution_01_07_amp.png}
    \label{fig:execution_01_07_time}
    \caption{70\% modulation index amplitude, shown at 3.52V}
\end{figure}

\begin{figure}[H]
    \centering
    \includegraphics[width=0.75\textwidth]{images/execution_01_07_freq.png}
    \label{fig:execution_01_07_frequency}
    \caption{70\% modulation index frequency, shown at 495Hz}
\end{figure}

The modulation index is then changed to 50\%.
\begin{figure}[H]
    \centering
    \includegraphics[width=0.75\textwidth]{images/execution_01_05_amp.png}
    \label{fig:execution_01_05_amplitude}
    \caption{50\% modulation index amplitude, shown at 2.48V}
\end{figure}
\begin{figure}[H]
    \centering
    \includegraphics[width=0.75\textwidth]{images/execution_01_05_freq.png}
    \label{fig:execution_01_05_frequency}
    \caption{50\% modulation index frequency, shown at 500Hz}
\end{figure}

The modulation index is then changed to 120\% to show an overmodulated signal.

\begin{figure}[H]
    \centering
    \includegraphics[width=0.75\textwidth]{images/execution_01_12_amp.png}
    \label{fig:execution_01_12_amplitude}
    \caption{120\% modulation index amplitude, shown at 6V}
\end{figure}

\section{Problem 2: AM Modulated Signals in Frequency Domain}

The same setup is used, with the AM modulation index now once more set to 70\%.

The amplitude modulated signal's frequency is then measured via the use of the oscilloscope's FFT function.

\begin{figure}[H]
    \centering
    \includegraphics[width=0.75\textwidth]{images/execution_02_07_fft.png}
    \label{fig:execution_02_07_fft}
    \caption{70\% modulation index FFT, with the peaks at 20kHz and 19.81kHz}
\end{figure}
\newpage
\section{Problem 3: Demodulation of a message signal}

The following circuit is assembled on the breadboard,
\begin{figure}[H]
    \centering
    \includegraphics[width=0.75\textwidth]{images/execution_03_circuit.png}
    \label{fig:execution_03_circuit}
    \caption{Demodulation circuit}
\end{figure}

The following settings are used for the oscilloscope:
\begin{itemize}
    \item Signal Shape = Sine
    \item Modulation = AM
    \item Carrier Frequency = 20KHz
    \item Carrier Amplitude = 10Vpp
    \item Modulation Frequency = 500Hz
    \item Modulation Index = 50\%
\end{itemize}

The AM modulated signal is displayed with the first-order filter output.
\begin{figure}[H]
    \centering
    \includegraphics[width=0.75\textwidth]{images/execution_03_first_order.png}
    \label{fig:execution_03_first_order}
    \caption{First order filter output}
\end{figure}
The AM modulated signal is then displayed with the third-order filter output.
\begin{figure}[H]
    \centering
    \includegraphics[width=0.75\textwidth]{images/execution_03_third_order.png}
    \label{fig:execution_03_third_order}
    \caption{Third order filter output}
\end{figure}

The amplitude of the signal produced by the third-order filter is then measured.

\begin{figure}[H]
    \centering
    \includegraphics[width=0.75\textwidth]{images/execution_03_third_order_amp.png}
    \label{fig:execution_03_third_order_amp}
    \caption{Third order filter amplitude, shown at 416mV}
\end{figure}

Furthermore, the FFT of the third-order filter output is taken, and the 20kHz component remains.



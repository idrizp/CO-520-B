\section{Prelab Code:}
\begin{verbatim}
%% Prelab:

% Prelab Problem 2:
Fs = 1e5;
t = 0:1/Fs:0.01;
f = 20E3; % 20 Khz

A_c = 5;

f_m = 5E2;
x = sin(2*pi*f_m*t);

f_c = 20E3;
y = A_c*(1+0.5*x).*cos(2*pi*f_c*t);

subplot(3, 2, 1);
plot(t, x);
title('Modulating Signal');
xlabel('Time(in s)');
ylabel('Voltage(in V)');

subplot(3, 2, 2);
plot(t, y);
title('Amplitude Modulated Signal');
xlabel('Time (in s)');
ylabel('Voltage (in V)');

% Frequency spectrum

N = length(y);
Y = fft(y);

spectrum = abs(Y/N);
spectrum_single = spectrum(1:N/2+1);
spectrum_single(1:end-1) = 2*spectrum_single(1:end-1);

F = Fs * (0:(N/2)) / N;
subplot(3, 2, 3);
plot(F, spectrum_single);
xlim([1E3, 5E4]);
title('Amplitude Modulated Signal Frequency Spectrum');

frequencies = logspace(3, 5, 100); % 100 Hz to 100 KHz
Wn = 1000/(f/2); % Normalized cutoff frequency

[b1, a1] = butter(1,Wn); % Butterworth filter of first order
[b3, a3] = butter(3, Wn); % Butterworth filter of third order

% Rectify the signal

% We can use an envelope detector for this, the simplest analogue to a
% diode would be the absolute value of the signal, so that's what will be
% used

rectified = abs(y);
filtered = filter(b1, a1, rectified);

% figure;
subplot(3, 2, 4);
plot(t, rectified);
title('First-Order Demodulated Signal');
xlabel('Time(in s)');
ylabel('Voltage(in V)');

filtered3 = filter(b3, a3, rectified);
subplot(3, 2, 5);
plot(t, filtered3);
title('Third-Order Demodulated Signal');
xlabel('Time(in s)');
ylabel('Voltage(in V)');

% Bode plot for the filters
figure;
freqs(b1, a1);
title("First Order Butterworth Filter Bode Plot");

figure;
freqs(b3, a3);
title("Third Order Butterworth Filter Bode Plot");
\end{verbatim}
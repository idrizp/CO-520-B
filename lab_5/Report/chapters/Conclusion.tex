The behavior of AM modulation was explored via the use of the function generator's ability to generate AM signals, and the behaviour was observed using the oscilloscope. Furthermore, a slope detection circuit was assembled on the breadboard in order to demodulate the AM signal. The difference between first and third order slope detection was observed. The FFT of the oscilloscope's resolution, in part, causes the error in trying to observe the frequency peaks and finding the true modulation index in the case of 70\% modulation provided by the function generator, where it was obtained from the determined frequency that it was 82\%. Artifacts in the de-modulated signal are due to the internal resistance of the inductors, capacitors, and also due to the other components not being ideal. The wires have their own resistance as well. Overmodulation was also observed, and the effect it has on the envelope of the signal.